\chapter{General Tools}
\section{Makefile}



\begin{lstlisting}
# These are variables that allow us to easily change compiler options
CC=g++
CFLAGS= -c -Wall

#Calling make all (or just make as that executes the first line by default) 
#will make make check the existence of prog  
#thus it will go to the next line where we defined what prog is and 
#what its dependencies are.

all: prog


#To get prog we ned main.o factorial.o and hello.o
#make will look up whether they are present and are up to date**********
#(it checks the last modified date, if the dependencies are newer it re-compiles)
#if not it will re-compile the required dependencies***********
#and finally when all the dependencies are up to date it will execute the 
#command provided in the next line after a "'''tab spacing'''"
#*********using spaces instead of tabs will not work.*************
#the same format is followed throughout

prog: main.o factorial.o hello.o
    $(CC) main.o factorial.o hello.o -o prog
    
    
main.o: main.cpp
    $(CC) $(CFLAGS) main.cpp
    
    
factorial.o: factorial.cpp
    $(CC) $(CFLAGS) factorial.cpp
    
    
hello.o: hello.cpp
    $(CC) $(CFLAGS) hello.cpp
    
#this is a way to make make do some cleaning
#make clean on being called will check for dependencies
#there are none so it will simply execute the provided command 
#which in this case will remove all the object files
#this same trick can be used to do things like building and 
#then copying copying the binary to some other location**************
#like in other machines connected to the current machine and so on


clean:
    rm -rf *.o 
    
\end{lstlisting}


\section{regex/regular expressions}

\href{https://codepen.io/jakealbaugh/post/regex-in-sublime-text}{This is fun and quick intro and has further references.}



\section{shell_basics}


\href{http://www.newthinktank.com/2016/06/shell-scripting-tutorial/}{Has well paced accompanying video as well.}



\begin{lstlisting}[language=bash,caption={basics of bash}]
I. Intro

1. A shell script contains commands that are executed as if you typed them in the terminal.

2. We'll be using Vim for this tutorial

	a. Install Vim : sudo apt-get install vim
	
	b. Vim Commands
	
		1. i : insert mode

		2. <ESC> : enter command mode
			i. w : Save / Don't Exit
			ii. wq : Save / Quit
			iii. q! : Quit / Discard Changes
			iv. w : Move to front of next word
			v. b : Move backwards to front of word
			vi. 0 : Move to start of line
			vii. $ : Move to end of line
			viii. G : Jump to last line
		
		3. Move around with arrows
		
		4. :set number : Displays line numbers
		
		5. :syntax on : Syntax Highlighting
		
		6. :set tabstop=2 : Spaces in tab
		
		7. :set autoindent : Indent new lines
		
		8. Save these in your home/~/.vimrc file
		
			a. Find out what vimrc file you are using with this command in Vim :echo $MYVIMRC

3. Hello World Script
# The #! shebang tells the system the interpreter to use for the script
#!/bin/bash
# Comment
echo 'Hello World' # Print the string to the screen

	a. To make it executable chmod 755 hello_world
	
	b. Execute with ./hello_world
	
	c. The numbers after chmod define who can do what with the file
	
	d. The numbers represent the Owner, the Group and Everyone else
	
	e. What the numbers mean
	
		1. 7 : Read, Write & Execute
		2. 6 : Read & Write
		3. 5 : Read & Execute
		4. 4 : Read Only
		5. 3 : Write & Execute
		6. 2 : Write Only
		7. 1 : Execute Only
		8. 0 : None
		
4. We define variables like this myName="Derek"

	a. The variable name starts with a letter or _ and then can also contain numbers
	
	b. The shell treats all variables as strings
	
	c. When declaring a variable you can't have whitespace on either side of the =
	
	d. 
	#!/bin/bash
	declare -r NUM1=5 # Declare a constant
	num2=4
	
	# Use arithmetic expansion for adding
	num3=$((NUM1+num2))
	num4=$((NUM1-num2))
	num5=$((NUM1*num2))
	num6=$((NUM1/num2))
	
	# Place variables in strings with $
	echo "5 + 4 = $num3"
	echo "5 - 4 = $num4"
	echo "5 * 4 = $num5"
	echo "5 / 4 = $num6"
	echo $(( 5**2 ))
	echo $(( 5%4 ))
	
	# Assignment operators allow for shorthand arithmetic 
	# +=, -=, *=, /=
	rand=5
	let rand+=4
	echo "$rand"
	
	# Shorthand increment and decrement
	echo "rand++ = $(( rand++ ))"
	echo "++rand = $(( ++rand ))"
	echo "rand-- = $(( rand-- ))"
	echo "--rand = $(( --rand ))"
	
	# Use Python to add floats
	num7=1.2
	num8=3.4
	num9=$(python -c "print $num7+$num8")
	echo $num9
	
	# You can print over multiple lines with a Here Script
	# cat prints a file or any string past to it
	cat << END
	This text
	prints on
	many lines
	END

 II. Functions
 
 	1. You can use functions to avoid the need to write duplicate code
 	
 	2. Delete all code in Vim with gg then dG
 	
 	3. #!/bin/bash
 	# Define function
 	getDate() {
 		
 		# Get current date and time
 		date
 		
 		# Return returns an exit status number between 0 - 255
 		return
 	}
 	
 	getDate
 	
 	# This is a global variable
 	name="Derek"
 	
 	# Local variable values aren't available outside of the function
 	demLocal() {
 		local name="Paul"
 		return
 	}
 	
 	demLocal
 	
 	echo "$name"
 	
 	# A function that receives 2 values and prints a sum
 	getSum() {
 	
 		# Attributes are retrieved by referring to $1, $2, etc.
 		local num3=$1
 		local num4=$2
 		
 		# Sum values
 		local sum=$((num3+num4))
 		
 		# Pass values back with echo
 		echo $sum
 	}
 	
 	num1=5
 	num2=6
 	
 	# You pass atributes by separating them with a space
 	# Surround function call with $() to get the return value
 	sum=$(getSum num1 num2)
 	echo "The sum is $sum"
 	
III. Conditionals / Input 

	1. 
	#!/bin/bash
	
	# You can use read to receive input which is stored in name
	# The p option says that we want to prompt with a string
  	read -p "What is your name? " name
  	echo "Hello $name"
  
  	read -p "How old are you? " age
  	 
  	# You place your condition with in []
  	# Include a space after [ and before ]
  	# Integer Comparisons: eq, ne, le, lt, ge, gt
  	if [ $age -ge 16 ]
  	then
  		echo "You can drive"
  	
  	# Check another condition
  	elif [ $age -eq 15 ]
  	then
  		echo "You can drive next year"
  		
  	# Executed by default
 	else
 	  echo "You can't drive"
 	  
 	# Closes the if statement
 	fi
 	
 	2. Extended integer test
 	#!/bin/bash
 	
 	read -p "Enter a number : " num
 	
 	if ((num == 10)); then
 		echo "Your number equals 10"
 	fi
 	
 	if ((num > 10)); then
 		echo "It is greater then 10"
 	else
 		echo "It is less then 10"
 	fi
 	
 	if (( ((num % 2)) == 0 )); then
 		echo " It is even"
 	fi
 	
 	# You can use logical operators like &&, || and !
 	if (( ((num > 0)) && ((num < 11)) )); then
 		echo "$num is between 1 and 10"
 	fi
 	
 	# && and || can be used as control structures
 	
 	# Create a file and then if that worked open it in Vim
 	touch samp_file && vim samp_file
 	
 	# If samp_dir doesn't exist make it
 	[ -d samp_dir ] || mkdir samp_dir
 	
 	# Delete file rm samp_file
 	# Delete directory rmdir samp_dir

	3. Testing strings
	#!/bin/bash
	str1=""
	str2="Sad"
	str3="Happy"
	
	# Test if a string is null
	if [ "$str1" ]; then
		echo "$str1 is not null"
	fi
	
	if [ -z "$str1" ]; then
		echo "str1 has no value"
	fi
	
	# Check for equality
	if [ "$str2" == "$str3" ]; then
		echo "$str2 equals $str3"
	elif [ "$str2" != "$str3" ]; then
		echo "$str2 is not equal to $str3"
	fi
	
	if [ "$str2" > "$str3" ]; then
		echo "$str2 is greater then $str3"
	elif [ "$str2" < "$str3" ]; then
		echo "$str2 is less then $str3"
	fi
	
	# Check the file test_file1 and test_file2
	file1="./test_file1"
	file2="./test_file2"
	
	if [ -e "$file1" ]; then
		echo "$file1 exists"
		
		if [ -f "$file1" ]; then
			echo "$file1 is a normal file"
		fi
		
		if [ -r "$file1" ]; then
			echo "$file1 is readable"
		fi
		
		if [ -w "$file1" ]; then
			echo "$file1 is writable"
		fi
		
		if [ -x "$file1" ]; then
			echo "$file1 is executable"
		fi
		
		if [ -d "$file1" ]; then
			echo "$file1 is a directory"
		fi
		
		if [ -L "$file1" ]; then
			echo "$file1 is a symbolic link"
		fi
		
		if [ -p "$file1" ]; then
			echo "$file1 is a named pipe"
		fi
		
		if [ -S "$file1" ]; then
			echo "$file1 is a network socket"
		fi
		
		if [ -G "$file1" ]; then
			echo "$file1 is owned by the group"
		fi
		
		if [ -O "$file1" ]; then
			echo "$file1 is owned by the userid"
		fi
		
	fi
	
	4. With extended test [[ ]] you can use Regular Expressions
	#!/bin/bash
	
	read -p "Validate Date : " date
	
	pat="^[0-9]{8}$"
	
	if [[ $date =~ $pat ]]; then
		echo "$date is valid"
	else
		echo "$date is not valid"
	fi
	
	5. # Read multiple values
	#!/bin/bash
	
	read -p "Enter 2 Numbers to Sum : " num1 num2
	
	sum=$((num1+num2))
	
	echo "$num1 + $num2 = $sum"
	
	# Hide the input with the s code
	read -sp "Enter the Secret Code" secret
	
	if [ "$secret" == "password" ]; then
		echo "Enter"
	else
		echo "Wrong Password"
	fi
	
	6. You can set what separates the values with IFS
	#!/bin/bash
	
	# Store the original value of IFS
	OIFS="$IFS"
	
	# Set what separates the input values
	IFS=","
	
	read -p "Enter 2 numbers to add separated by a comma" num1 num2
	
	# Use the parameter expansion ${} to substitute any whitespace
	# with nothing
	num1=${num1//[[:blank:]]/}
	num2=${num2//[[:blank:]]/}

	sum=$((num1+num2))
	
	echo "$num1 + $num2 = $sum"
	
	# Reset IFS to the original value
	IFS="$OIFS"
	
	# Parameter expansion allows you to do this
	name="Derek"
	echo "${name}'s Toy"
	
	# The search and replace allows this
	samp_string="The dog climbed the tree"
	echo "${samp_string//dog/cat}"
	
	# You can assign a default value if it doesn't exist
	echo "I am ${name:-Derek}"
	
	# This uses the default if it doesn't exist and assigns the value
	# to the variable
	echo "I am ${name:=Derek}"
	echo $name
	
	7. Use case to when it makes more sense then if
	#!/bin/bash
	
	read -p "How old are you : " age
	
	# Check the value of age
	case $age in
	
	# Match numbers 0 - 4
	[0-4]) 
		echo "To young for school"
		;; # Stop checking further
		
	# Match only 5
	5)
		echo "Go to kindergarten"
		;;
		
	# Check 6 - 18
	[6-9]|1[0-8])
		grade=$((age-5))
		echo "Go to grade $grade"
		;;
		
	# Default action
	*)
		echo "You are to old for school"
		;;
	esac # End case
	
	8. Ternary Operator performs different actions based on a condition
	#!/bin/bash
	can_vote=0
	age=18
	
	((age>=18?(can_vote=1):(can_vote=0)))
	echo "Can Vote : $can_vote"
	
	
IV. Parameter Expansions and Strings

	1. Strings
	#!/bin/bash
	
	rand_str="A random string"
	
	# Get string length
	echo "String Length : ${#rand_str}"
	
	# Get string slice starting at index (0 index)
	echo "${rand_str:2}"
	
	# Get string with starting and ending index
	echo "${rand_str:2:7}"
	
	# Return whats left after A
	echo "${rand_str#*A }"

V. Looping

	1. While Loop
	#!/bin/bash
	
	num=1
	
	while [ $num -le 10 ]; do
		echo $num
		num=$((num + 1))
	done
	
	2. Continue and Break
	#!/bin/bash
	
	num=1
	
	while [ $num -le 20 ]; do
	
		# Don't print evens
		if (( ((num % 2)) == 0 )); then
 			num=$((num + 1))
 			continue
 		fi
 		
 		# Jump out of the loop with break
 		if ((num >= 15)); then
 			break
 		fi
 		
		echo $num
		num=$((num + 1))
	done
	
	3. Until loops until the loop is true
	#!/bin/bash
	
	num=1
	
	until [ $num -gt 10 ]; do
		echo $num
		num=$((num + 1))
	done
	
	4. Use read and a loop to output file info
	#!/bin/bash
  	while read avg rbis hrs; do
  	
  		# printf allows you to use \n
  		printf "Avg: ${avg}\nRBIs: ${rbis}\nHRs: ${hrs}\n"
  		
  	# Pipe data into the while loop
  	done < barry_bonds.txt
  	
  	5. There are many for loop options. Here is the C form.
  	#!/bin/bash
  	for (( i=0; i <= 10; i=i+1 )); do
  		echo $i
  	done
  	
  	6. We can cycle through ranges
  	#!/bin/bash
  	for i in {A..Z}; do
  		echo $i
  	done
  	
  	7.
  	
VI. Arrays

	1. Bash arrays can only have one dimension and indexes start at 0
	
	2. Messing with arrays
	#!/bin/bash
	
	# Create an array
	fav_nums=(3.14 2.718 .57721 4.6692)
	
	echo "Pi : ${fav_nums[0]}"
	
	# Add value to array
	fav_nums[4]=1.618
	
	echo "GR : ${fav_nums[4]}"
	
	# Add group of values to array
	fav_nums+=(1 7)
	
	# Output all array values
	for i in ${fav_nums[*]}; do
		echo $i;
	done
	
	# Output indexes
	for i in ${!fav_nums[@]}; do
		echo $i;
	done
	
	# Get number of items in array
	echo "Array Length : ${#fav_nums[@]}"
	
	# Get length of array element
	echo "Index 3 length : ${#fav_nums[3]}"
	
	# Sort an array
	sorted_nums=($(for i in "${fav_nums[@]}"; do
		echo $i;
	done | sort))
	
	for i in ${sorted_nums[*]}; do
		echo $i;
	done
	
	# Delete array element
	unset 'sorted_nums[1]'
	
	# Delete Array
	unset sorted_nums

	
VII. Positional Parameters

	1. Positional parameters are variables that can store data on the command line in variable names 0 - 9
	
		a. $0 always contains the path to the executed script
		
		b. You can access names past 9 by using parameter expansion like this ${10}
		
	2. Add all numbers on the command line
	#!/bin/bash
	
	# Print the first argument
	echo "1st Argument : $1"
	
	sum=0
	
	# $# tells you the number of arguments
	while [[ $# -gt 0 ]]; do
	
		# Get the first argument
		num=$1
		sum=$((sum + num))
		
		# shift moves the value of $2 into $1 until none are left
		# The value of $# decrements as well
		shift
	done
	
	echo "Sum : $sum"
\end{lstlisting}