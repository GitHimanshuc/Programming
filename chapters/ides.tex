\chapter{IDES}

\section{VSCode}

\subsection{Some important files}

c-cpp-properties.json contains all the configuration related things, like what is the location of libraries, the standard of compilers to use their path and so on. It is genrated by default once the a project is started.

\bigbreak

\begin{lstlisting}
{
    "configurations": [
        {
            "name": "Linux",
            "includePath": [
                "${workspaceFolder}/**",
                "/usr/include/mpi"
            ],
            "defines": [],
            "compilerPath": "/usr/bin/clang",
            "cStandard": "c11",
            "cppStandard": "c++17",
            "intelliSenseMode": "clang-x64"
        }
    ],
    "version": 4
}
\end{lstlisting}


\bigbreak
\noindent\rule{\textwidth}{1pt}
\bigbreak
\noindent
settings.json contains general information about the behaviour of the ide. It also contains the path to various soft wares like Julia. It also has a workspace settings tab that basically overwrites the default system-wide settings for that particular project.\\

\begin{lstlisting}
{
    "python.pythonPath": "/home/himanshu/miniconda3",
    "julia.executablePath": "/home/himanshu/Softwares/julia-1.0.0/bin/julia",
    "julia.enableTelemetry": true,
    "editor.multiCursorModifier": "ctrlCmd",
    "editor.formatOnPaste": true,
    "editor.snippetSuggestions": "top",
    "workbench.colorTheme": "Monokai",
    "python.autoComplete.addBrackets": true,
    "python.formatting.provider": "autopep8",
    "editor.formatOnSave": true,
    "julia.runlinter": false,
    "[julia]": {}
}
\end{lstlisting}

\bigbreak
\noindent\rule{\textwidth}{1pt}
\bigbreak
\noindent
tasks.json , if not present then create by using Ctlr-Shift-P follwed by Configure Tasks.
This is the most important file in the sense that it contains all the compiler run-time options like debugging.

\begin{lstlisting}
{
    // See https://go.microsoft.com/fwlink/?LinkId=733558
    // for the documentation about the tasks.json format
    "version": "2.0.0",
    "tasks": [
        {
            "label": "build hello world",
            "type": "shell",
            "command": "mpicc",
            "args": [
                "-g",
                "Main.c",
                "-lm"
            ],
            "group": {
                "kind": "build",
                "isDefault": true
            }
        }
    ]
}
\end{lstlisting}


\bigbreak
\noindent\rule{\textwidth}{1pt}
\bigbreak


\noindent
launch.json this file as far as I can guess is related to debugging.
\\
\\
If it is not present then go to the debug menu and after selecting the appropriate debugger use the green run button then give the appropriate destination of the compiled code.

\begin{lstlisting}
{
    // Use IntelliSense to learn about possible attributes.
    // Hover to view descriptions of existing attributes.
    // For more information, visit: https://go.microsoft.com/fwlink/?linkid=830387
    "version": "0.2.0",
    "configurations": [
        {
            "name": "(gdb) Launch",
            "type": "cppdbg",
            "request": "launch",
            "program": "/home/himanshu/Desktop/final-undergraduate_project/2D_code/a.out",
            "args": [],
            "stopAtEntry": false,
            "cwd": "/home/himanshu/Desktop/final-undergraduate_project/2D_code/",
            "environment": [],
            "externalConsole": true,
            "MIMode": "gdb",
            "setupCommands": [
                {
                    "description": "Enable pretty-printing for gdb",
                    "text": "-enable-pretty-printing",
                    "ignoreFailures": true
                }
            ]
        }
    ]
\end{lstlisting}

\subsection{c/cpp-debugging}

Press the debug button on the sidebar. Can set breakpoints using the red dots that appear when the cursor is on the line-numbering. Click on the red-dots to get the option to edit them, this allows you to set conditions when the break-points will activate. Then run and enjoy. This can handle multi-threaded codes.



\section{vim}

Why i wasted almost 5 hours trying to use it is a mystery to me. But anyways things are more or less configured as per my taste - just simple auto completion and bracket closing moving a line up and down etc.

\begin{lstlisting}
set nocompatible              " required
filetype off                  " required


set rtp+=~/.vim/bundle/Vundle.vim
call vundle#begin()


Plugin 'gmarik/Vundle.vim'
Plugin 'tmhedberg/SimpylFold'
Plugin 'vim-scripts/indentpython.vim'
Bundle 'Valloric/YouCompleteMe'
Plugin 'vim-syntastic/syntastic'
Plugin 'nvie/vim-flake8'

Plugin 'scrooloose/nerdtree'


"Ctlr-P seaching for everything
Plugin 'kien/ctrlp.vim'

"Basic git 
Plugin 'tpope/vim-fugitive'

call vundle#end()            " required
filetype plugin indent on    " required


"split navigations
nnoremap <C-J> <C-W><C-J>
nnoremap <C-K> <C-W><C-K>
nnoremap <C-L> <C-W><C-L>
nnoremap <C-H> <C-W><C-H>

set splitbelow
set splitright


" This is so that our spacing and all follows PIP8 standard.
au BufNewFile,BufRead *.py
    \ set tabstop=4
    \ set softtabstop=4
    \ set shiftwidth=4
    \ set textwidth=79
    \ set expandtab
    \ set autoindent
    \ set fileformat=unix

set encoding=utf-8

" This  will mark in red any extra white spaces that are present.
au BufRead,BufNewFile *.py,*.pyw,*.c,*.h match BadWhitespace /\s\+$/

" auto-complete goes away when we are done. The next line maps space-g to goto
" definition
let g:ycm_autoclose_preview_window_after_completion=1
map <leader>g  :YcmCompleter GoToDefinitionElseDeclaration<CR>


"python with virtualenv support
"py << EOF
"import os
"import sys
"if 'VIRTUAL_ENV' in os.environ:
"  project_base_dir = os.environ['VIRTUAL_ENV']
"  activate_this = os.path.join(project_base_dir, 'bin/activate_this.py')
"  execfile(activate_this, dict(__file__=activate_this))
"EOF

"Pretty looking python code

let python_highlight_all=1
syntax on



"Line numbers
set number relativenumber

"Bracket completion
inoremap " ""<left>
inoremap ' ''<left>
inoremap ( ()<left>
inoremap [ []<left>
inoremap { {}<left>
inoremap {<CR> {<CR>}<Esc>O<Tab>
inoremap {;<CR> {<CR>};<Esc>O
"map {<CR> :{<CR>{<CR>}<Esc>O


"Moving lines up or down using Alt-j/k
nnoremap <A-j> :m .+1<CR>==
nnoremap <A-k> :m .-2<CR>==
inoremap <A-j> <Esc>:m .+1<CR>==gi
inoremap <A-k> <Esc>:m .-2<CR>==gi
vnoremap <A-j> :m '>+1<CR>gv=gv
vnoremap <A-k> :m '<-2<CR>gv=gv


"Some weird thing that has to done for Alt keymaps to work. Something to do
"with meta keys and all??????????????
"
"Breaks a lot of other things, like yy followed by p was not working
" Somehow Things are working now????????????
"
"for i in range(97,122)
"  let c = nr2char(i)
"  exec "map \e".c." <M-".c.">"
"  exec "map! \e".c." <M-".c.">"
"endfor

\end{lstlisting}


\subsection{some_useful_key_bindings}
\textbf{Substitute multiple words/Will not work if substitution has space in between}

\begin{enumerate}
    \item /(type the word to be exchanged)
    \item press enter,then <cgn>
    \item type in the new word
    \item press <esc>
    \item . the required number of times
\end{enumerate}

\textbf{Almost like multiple cursors}
\href{https://stackoverflow.com/questions/11784408/vim-multiline-editing-like-in-sublimetext}{Link of stack overflow answer.}


\textbf{Creating a new file from inside vim}

:e (file path)



\textbf{Wrapping text without without breaking words}
:set linebreak

\subsection{Spell check}

\begin{enumerate}
    \item type :set spell to start spell check
    \item wrong words are highlighted in red
    \item to go through the wrong words use [s or ]s
    \item to add a word move cursor to that word and press zg
    \item to remove a word form spell file use zug
    \item \href{http://vimdoc.sourceforge.net/htmldoc/spell.html}{docs}
\end{enumerate}

